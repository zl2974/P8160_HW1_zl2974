% Options for packages loaded elsewhere
\PassOptionsToPackage{unicode}{hyperref}
\PassOptionsToPackage{hyphens}{url}
%
\documentclass[
]{article}
\usepackage{amsmath,amssymb}
\usepackage{lmodern}
\usepackage{ifxetex,ifluatex}
\ifnum 0\ifxetex 1\fi\ifluatex 1\fi=0 % if pdftex
  \usepackage[T1]{fontenc}
  \usepackage[utf8]{inputenc}
  \usepackage{textcomp} % provide euro and other symbols
\else % if luatex or xetex
  \usepackage{unicode-math}
  \defaultfontfeatures{Scale=MatchLowercase}
  \defaultfontfeatures[\rmfamily]{Ligatures=TeX,Scale=1}
\fi
% Use upquote if available, for straight quotes in verbatim environments
\IfFileExists{upquote.sty}{\usepackage{upquote}}{}
\IfFileExists{microtype.sty}{% use microtype if available
  \usepackage[]{microtype}
  \UseMicrotypeSet[protrusion]{basicmath} % disable protrusion for tt fonts
}{}
\makeatletter
\@ifundefined{KOMAClassName}{% if non-KOMA class
  \IfFileExists{parskip.sty}{%
    \usepackage{parskip}
  }{% else
    \setlength{\parindent}{0pt}
    \setlength{\parskip}{6pt plus 2pt minus 1pt}}
}{% if KOMA class
  \KOMAoptions{parskip=half}}
\makeatother
\usepackage{xcolor}
\IfFileExists{xurl.sty}{\usepackage{xurl}}{} % add URL line breaks if available
\IfFileExists{bookmark.sty}{\usepackage{bookmark}}{\usepackage{hyperref}}
\hypersetup{
  pdftitle={Homework on optimization algorithms.},
  hidelinks,
  pdfcreator={LaTeX via pandoc}}
\urlstyle{same} % disable monospaced font for URLs
\usepackage[margin=1in]{geometry}
\usepackage{color}
\usepackage{fancyvrb}
\newcommand{\VerbBar}{|}
\newcommand{\VERB}{\Verb[commandchars=\\\{\}]}
\DefineVerbatimEnvironment{Highlighting}{Verbatim}{commandchars=\\\{\}}
% Add ',fontsize=\small' for more characters per line
\usepackage{framed}
\definecolor{shadecolor}{RGB}{248,248,248}
\newenvironment{Shaded}{\begin{snugshade}}{\end{snugshade}}
\newcommand{\AlertTok}[1]{\textcolor[rgb]{0.94,0.16,0.16}{#1}}
\newcommand{\AnnotationTok}[1]{\textcolor[rgb]{0.56,0.35,0.01}{\textbf{\textit{#1}}}}
\newcommand{\AttributeTok}[1]{\textcolor[rgb]{0.77,0.63,0.00}{#1}}
\newcommand{\BaseNTok}[1]{\textcolor[rgb]{0.00,0.00,0.81}{#1}}
\newcommand{\BuiltInTok}[1]{#1}
\newcommand{\CharTok}[1]{\textcolor[rgb]{0.31,0.60,0.02}{#1}}
\newcommand{\CommentTok}[1]{\textcolor[rgb]{0.56,0.35,0.01}{\textit{#1}}}
\newcommand{\CommentVarTok}[1]{\textcolor[rgb]{0.56,0.35,0.01}{\textbf{\textit{#1}}}}
\newcommand{\ConstantTok}[1]{\textcolor[rgb]{0.00,0.00,0.00}{#1}}
\newcommand{\ControlFlowTok}[1]{\textcolor[rgb]{0.13,0.29,0.53}{\textbf{#1}}}
\newcommand{\DataTypeTok}[1]{\textcolor[rgb]{0.13,0.29,0.53}{#1}}
\newcommand{\DecValTok}[1]{\textcolor[rgb]{0.00,0.00,0.81}{#1}}
\newcommand{\DocumentationTok}[1]{\textcolor[rgb]{0.56,0.35,0.01}{\textbf{\textit{#1}}}}
\newcommand{\ErrorTok}[1]{\textcolor[rgb]{0.64,0.00,0.00}{\textbf{#1}}}
\newcommand{\ExtensionTok}[1]{#1}
\newcommand{\FloatTok}[1]{\textcolor[rgb]{0.00,0.00,0.81}{#1}}
\newcommand{\FunctionTok}[1]{\textcolor[rgb]{0.00,0.00,0.00}{#1}}
\newcommand{\ImportTok}[1]{#1}
\newcommand{\InformationTok}[1]{\textcolor[rgb]{0.56,0.35,0.01}{\textbf{\textit{#1}}}}
\newcommand{\KeywordTok}[1]{\textcolor[rgb]{0.13,0.29,0.53}{\textbf{#1}}}
\newcommand{\NormalTok}[1]{#1}
\newcommand{\OperatorTok}[1]{\textcolor[rgb]{0.81,0.36,0.00}{\textbf{#1}}}
\newcommand{\OtherTok}[1]{\textcolor[rgb]{0.56,0.35,0.01}{#1}}
\newcommand{\PreprocessorTok}[1]{\textcolor[rgb]{0.56,0.35,0.01}{\textit{#1}}}
\newcommand{\RegionMarkerTok}[1]{#1}
\newcommand{\SpecialCharTok}[1]{\textcolor[rgb]{0.00,0.00,0.00}{#1}}
\newcommand{\SpecialStringTok}[1]{\textcolor[rgb]{0.31,0.60,0.02}{#1}}
\newcommand{\StringTok}[1]{\textcolor[rgb]{0.31,0.60,0.02}{#1}}
\newcommand{\VariableTok}[1]{\textcolor[rgb]{0.00,0.00,0.00}{#1}}
\newcommand{\VerbatimStringTok}[1]{\textcolor[rgb]{0.31,0.60,0.02}{#1}}
\newcommand{\WarningTok}[1]{\textcolor[rgb]{0.56,0.35,0.01}{\textbf{\textit{#1}}}}
\usepackage{graphicx}
\makeatletter
\def\maxwidth{\ifdim\Gin@nat@width>\linewidth\linewidth\else\Gin@nat@width\fi}
\def\maxheight{\ifdim\Gin@nat@height>\textheight\textheight\else\Gin@nat@height\fi}
\makeatother
% Scale images if necessary, so that they will not overflow the page
% margins by default, and it is still possible to overwrite the defaults
% using explicit options in \includegraphics[width, height, ...]{}
\setkeys{Gin}{width=\maxwidth,height=\maxheight,keepaspectratio}
% Set default figure placement to htbp
\makeatletter
\def\fps@figure{htbp}
\makeatother
\setlength{\emergencystretch}{3em} % prevent overfull lines
\providecommand{\tightlist}{%
  \setlength{\itemsep}{0pt}\setlength{\parskip}{0pt}}
\setcounter{secnumdepth}{-\maxdimen} % remove section numbering
\ifluatex
  \usepackage{selnolig}  % disable illegal ligatures
\fi

\title{Homework on optimization algorithms.}
\author{}
\date{\vspace{-2.5em}P8160 Advanced Statistical Computing}

\begin{document}
\maketitle

\hypertarget{problem-1}{%
\subsection{Problem 1:}\label{problem-1}}

Design an optmization algorithm to find the minimum of the continuously
differentiable function \[f(x) =-e^{-x}\sin(x)\] on the closed interval
\([0,1.5]\). Write out your algorithm and implement it into \textbf{R}.

\hypertarget{answer-your-answer-starts-here}{%
\section{Answer: your answer starts
here\ldots{}}\label{answer-your-answer-starts-here}}

To find the minimum of a continuously function,we first make some
changes to the function let

\[g(x) = e^{x}\sin(x)\]

and instead find the maximum of g(x).

The gradient of \(g(x)\) is : \[\nabla g(x) = e^x(\sin(x)+\cos(x))\]

\begin{Shaded}
\begin{Highlighting}[]
\FunctionTok{ggplot}\NormalTok{(}\FunctionTok{tibble}\NormalTok{(}\AttributeTok{x =} \FunctionTok{seq}\NormalTok{(}\DecValTok{0}\NormalTok{,}\FloatTok{1.5}\NormalTok{,}\AttributeTok{length =} \DecValTok{10}\NormalTok{)),}\FunctionTok{aes}\NormalTok{(x))}\SpecialCharTok{+}
  \FunctionTok{geom\_function}\NormalTok{(}\AttributeTok{fun =} \ControlFlowTok{function}\NormalTok{(x) }\FunctionTok{exp}\NormalTok{(x)}\SpecialCharTok{*}\NormalTok{(}\FunctionTok{sin}\NormalTok{(x)}\SpecialCharTok{+}\FunctionTok{cos}\NormalTok{(x)))}
\end{Highlighting}
\end{Shaded}

and the Hessian is: \[\nabla^2g(x) = 2e^x\cos(x)\]

\begin{Shaded}
\begin{Highlighting}[]
\FunctionTok{plot}\NormalTok{(}\ControlFlowTok{function}\NormalTok{(x) }\DecValTok{2}\SpecialCharTok{*}\FunctionTok{exp}\NormalTok{(x)}\SpecialCharTok{*}\FunctionTok{cos}\NormalTok{(x),}\AttributeTok{xlim =} \FunctionTok{c}\NormalTok{(}\DecValTok{0}\NormalTok{,}\FloatTok{1.5}\NormalTok{))}
\end{Highlighting}
\end{Shaded}

the hessian is greater than 0 everywhere in {[}0,1.5{]}, so we can't use
Newton method.

\begin{Shaded}
\begin{Highlighting}[]
\NormalTok{goose\_egg }\OtherTok{=} 
  \ControlFlowTok{function}\NormalTok{(}
\NormalTok{    fun,}
    \AttributeTok{left =} \ConstantTok{NULL}\NormalTok{,}
    \AttributeTok{right =} \ConstantTok{NULL}\NormalTok{,}
    \AttributeTok{range =} \ConstantTok{NULL}\NormalTok{,}
    \AttributeTok{ratio =} \FloatTok{0.618}\NormalTok{,}
    \AttributeTok{tol =} \FloatTok{10e{-}4}\NormalTok{,}
\NormalTok{    ...}
\NormalTok{  )\{}
    \ControlFlowTok{if}\NormalTok{ (}\SpecialCharTok{!}\FunctionTok{any}\NormalTok{(left,right))\{}
\NormalTok{      left }\OtherTok{=}\NormalTok{ range[}\DecValTok{1}\NormalTok{]}
\NormalTok{      right }\OtherTok{=}\NormalTok{ range[}\DecValTok{2}\NormalTok{]}
\NormalTok{    \}}
    
\NormalTok{    mid\_1 }\OtherTok{=}\NormalTok{ left }\SpecialCharTok{+}\NormalTok{ ratio}\SpecialCharTok{*}\NormalTok{(right }\SpecialCharTok{{-}}\NormalTok{ left)}
    
\NormalTok{    f\_mid\_1 }\OtherTok{=} \FunctionTok{fun}\NormalTok{(mid\_1)}
    
\NormalTok{    mid\_2 }\OtherTok{=}\NormalTok{ mid\_1 }\SpecialCharTok{+}\NormalTok{ ratio}\SpecialCharTok{*}\NormalTok{(right}\SpecialCharTok{{-}}\NormalTok{mid\_1)}
    
\NormalTok{    f\_mid\_2 }\OtherTok{=} \FunctionTok{fun}\NormalTok{(mid\_2)}
    
\NormalTok{    f\_left }\OtherTok{=} \FunctionTok{fun}\NormalTok{(left)}
    
\NormalTok{    f\_right }\OtherTok{=} \FunctionTok{fun}\NormalTok{(right)}
    
\NormalTok{    i }\OtherTok{=} \DecValTok{1}
    
    \ControlFlowTok{while}\NormalTok{ (}\FunctionTok{abs}\NormalTok{(f\_left }\SpecialCharTok{{-}}\NormalTok{ f\_right)}\SpecialCharTok{\textgreater{}}\NormalTok{tol }\SpecialCharTok{\&\&}\NormalTok{ i}\SpecialCharTok{\textless{}}\DecValTok{1000}\NormalTok{)\{}
\NormalTok{      i }\OtherTok{=}\NormalTok{ i }\SpecialCharTok{+} \DecValTok{1}
      \ControlFlowTok{if}\NormalTok{ (f\_mid\_1 }\SpecialCharTok{\textless{}}\NormalTok{ f\_mid\_2) \{}
\NormalTok{        f\_left }\OtherTok{=}\NormalTok{ f\_mid\_1}
\NormalTok{        left }\OtherTok{=}\NormalTok{ mid\_1}
\NormalTok{      \} }\ControlFlowTok{else}\NormalTok{ \{}
\NormalTok{        f\_right }\OtherTok{=}\NormalTok{ f\_mid\_2}
\NormalTok{        right }\OtherTok{=}\NormalTok{ mid\_2}
\NormalTok{      \}}
\NormalTok{      mid\_1 }\OtherTok{=}\NormalTok{ left }\SpecialCharTok{+}\NormalTok{ ratio }\SpecialCharTok{*}\NormalTok{ (right }\SpecialCharTok{{-}}\NormalTok{ left)}
\NormalTok{      f\_mid\_1 }\OtherTok{=} \FunctionTok{fun}\NormalTok{(mid\_1)}
\NormalTok{      mid\_2 }\OtherTok{=}\NormalTok{ mid\_1 }\SpecialCharTok{+}\NormalTok{ ratio }\SpecialCharTok{*}\NormalTok{ (right }\SpecialCharTok{{-}}\NormalTok{ mid\_1)}
\NormalTok{      f\_mid\_2 }\OtherTok{=} \FunctionTok{fun}\NormalTok{(mid\_2)}
\NormalTok{    \}}
    \FunctionTok{return}\NormalTok{(}\FunctionTok{mean}\NormalTok{(mid\_1,mid\_2))}
\NormalTok{  \}}
\end{Highlighting}
\end{Shaded}

\begin{Shaded}
\begin{Highlighting}[]
\NormalTok{x\_max }\OtherTok{=} \FunctionTok{goose\_egg}\NormalTok{(}\ControlFlowTok{function}\NormalTok{(x) }\FunctionTok{exp}\NormalTok{(x)}\SpecialCharTok{*}\FunctionTok{sin}\NormalTok{(x),}\AttributeTok{range =} \FunctionTok{c}\NormalTok{(}\DecValTok{0}\NormalTok{,}\FloatTok{1.5}\NormalTok{))}

\FunctionTok{print}\NormalTok{(x\_max)}

\FunctionTok{plot}\NormalTok{(x\_max,}\SpecialCharTok{{-}}\FunctionTok{exp}\NormalTok{(x\_max)}\SpecialCharTok{*}\FunctionTok{sin}\NormalTok{(x\_max))}

\FunctionTok{plot}\NormalTok{(}\ControlFlowTok{function}\NormalTok{(x) \{}\SpecialCharTok{{-}}\FunctionTok{exp}\NormalTok{(x)}\SpecialCharTok{*}\FunctionTok{sin}\NormalTok{(x)\}, }\AttributeTok{xlim =} \FunctionTok{c}\NormalTok{(}\DecValTok{0}\NormalTok{,}\FloatTok{1.5}\NormalTok{), }\AttributeTok{add =}\NormalTok{ T)}
\end{Highlighting}
\end{Shaded}

\hypertarget{problem-2}{%
\subsection{Problem 2:}\label{problem-2}}

The Poisson distribution, written as\\
\[P(Y=y) = \frac{\lambda^y e^{-\lambda}}{y!}\] for \(\lambda > 0\), is
often used to model ``count'\,' data --- e.g., the number of events in a
given time period.

A Poisson regression model states that
\[Y_i \sim \textrm{Poisson}(\lambda_i),\] where
\[\log \lambda_i = \alpha + \beta x_i \] for some explanatory variable
\(x_i\). The question is how to estimate \(\alpha\) and \(\beta\) given
a set of independent data
\((x_1, Y_1), (x_2, Y_2), \ldots, (x_n, Y_n)\).

\begin{enumerate}
\item Generate a random sample $(x_i, Y_i)$ with $n=500$ from the Possion regression model above. 
You can choose the true parameters $(\alpha,\beta)$ and the distribution of $X$.

\item Write out the likelihood of your simulated data, and its Gradient and Hessian functions. 

\item  Develop a modify Newton-Raphson algorithm that allows the step-halving and re-direction steps
to ensure ascent directions and monotone-increasing properties. 

\item Write down your algorithm and implement it in R to estimate $\alpha$ and $\beta$ from your simulated data.
\end{enumerate}

\hypertarget{answer-your-answer-starts-here-1}{%
\section{Answer: your answer starts
here\ldots{}}\label{answer-your-answer-starts-here-1}}

\hypertarget{section}{%
\subsection{2.1}\label{section}}

\begin{Shaded}
\begin{Highlighting}[]
\BuiltInTok{print}\NormalTok{(}\StringTok{"hello world"}\NormalTok{)}
\end{Highlighting}
\end{Shaded}

\begin{Shaded}
\begin{Highlighting}[]
\NormalTok{X }\OtherTok{=} \FunctionTok{rbind}\NormalTok{(}\FunctionTok{rep}\NormalTok{(}\DecValTok{1}\NormalTok{,}\DecValTok{500}\NormalTok{),}\FunctionTok{rnorm}\NormalTok{(}\DecValTok{500}\NormalTok{))}
\NormalTok{Beta }\OtherTok{=} \FunctionTok{runif}\NormalTok{(}\DecValTok{2}\NormalTok{,}\DecValTok{1}\NormalTok{,}\DecValTok{10}\NormalTok{)}
\NormalTok{sigma }\OtherTok{=} \FunctionTok{rnorm}\NormalTok{(}\DecValTok{500}\NormalTok{,}\DecValTok{0}\NormalTok{,}\FloatTok{0.5}\NormalTok{)}
\NormalTok{lambda }\OtherTok{=} \FunctionTok{exp}\NormalTok{(}\FunctionTok{t}\NormalTok{(X)}\SpecialCharTok{\%*\%}\NormalTok{Beta }\SpecialCharTok{+}\NormalTok{ sigma)}
\NormalTok{Y }\OtherTok{=} \FunctionTok{map}\NormalTok{(lambda,}\SpecialCharTok{\textasciitilde{}}\FunctionTok{rpois}\NormalTok{(}\DecValTok{1}\NormalTok{,.x)) }\SpecialCharTok{\%\textgreater{}\%} \FunctionTok{unlist}\NormalTok{()}
\end{Highlighting}
\end{Shaded}

\hypertarget{section-1}{%
\subsection{2.2}\label{section-1}}

\begin{itemize}
\item
  The log-likelihood of Poisson distribution is
  \[l(Y;\lambda) = \sum \{y*log(\lambda) -\lambda - log(y!)\}\] OR
  \[l(Y;\alpha,\beta) = \sum \{y*(\alpha+x\beta) -exp(\alpha+x\beta) - log(y!)\}\]
\item
  The Score funtion is
  \[\nabla(Y;\lambda) = \frac{\partial}{\partial\lambda}l(Y;\lambda) = \sum\{\frac{y}{\lambda}-1\}\]
  OR
  \[\nabla(Y;\alpha,\beta) = \frac{\partial}{\partial\lambda}l(Y;\lambda) = (\sum\{y-exp(\alpha+x\beta)\},\sum\{y*x-x*exp(\alpha+x\beta)\})\]
\item
  The hessian is
  \[\nabla^2(Y;\lambda) = \frac{\partial^2}{\partial\lambda^2}l(Y;\lambda) = 
  -\sum\{\frac{y}{\lambda^2}+1\}=\] \textbackslash begin\{pmatrix\}
  \sum{-exp(\alpha+x\beta)} \& \sum{-x*exp(\alpha+x\beta)}
  \textbackslash{} \sum{-x*exp(\alpha+x\beta)} \&
  \sum{-x^2*exp(\alpha+\beta)} \textbackslash end\{\pmatrix\} which is
  negative defined everywhere.
\end{itemize}

\begin{Shaded}
\begin{Highlighting}[]
\NormalTok{Poisson }\OtherTok{=}
  \ControlFlowTok{function}\NormalTok{(x,}
\NormalTok{           lambda) \{}
\NormalTok{    log\_likelihood }\OtherTok{=} \FunctionTok{sum}\NormalTok{(x }\SpecialCharTok{*} \FunctionTok{log}\NormalTok{(lambda) }\SpecialCharTok{{-}}\NormalTok{ lambda }\SpecialCharTok{{-}} \FunctionTok{log}\NormalTok{(}\FunctionTok{factorial}\NormalTok{(x)))}
    
\NormalTok{    gradient }\OtherTok{=} \FunctionTok{sum}\NormalTok{(x }\SpecialCharTok{/}\NormalTok{ lambda }\SpecialCharTok{{-}} \DecValTok{1}\NormalTok{)}
    
\NormalTok{    hessian }\OtherTok{=}\NormalTok{ (}\SpecialCharTok{{-}}\DecValTok{1}\NormalTok{) }\SpecialCharTok{*} \FunctionTok{sum}\NormalTok{(x }\SpecialCharTok{/}\NormalTok{ lambda }\SpecialCharTok{{-}} \DecValTok{1}\NormalTok{)}
    
    \FunctionTok{return}\NormalTok{(}\FunctionTok{list}\NormalTok{(}
      \AttributeTok{log\_likelihood =}\NormalTok{ log\_likelihood,}
      \AttributeTok{gradient =}\NormalTok{ gradient,}
      \AttributeTok{hessian =}\NormalTok{ hessian}
\NormalTok{    ))}
\NormalTok{  \}}
\end{Highlighting}
\end{Shaded}

\hypertarget{section-2}{%
\subsection{2.3}\label{section-2}}

the Newton method updating is:
\[\nabla g(x_{k+1}) = \nabla g(x_k) + \eta*\nabla^2 g(x_k)(x_{k+1}-x_k)\]
where \(\eta\) is the step size that ensure
\(\nabla g(x_{k+1}) > \nabla g(x_k)\)

\begin{Shaded}
\begin{Highlighting}[]
\CommentTok{\#Develop a modify Newton{-}Raphson algorithm that allows the }
\CommentTok{\#step{-}halving and }
\CommentTok{\#re{-}direction steps}
\CommentTok{\#to ensure ascent directions and monotone{-}increasing properties. }

\NormalTok{newton\_update }\OtherTok{=}
  \ControlFlowTok{function}\NormalTok{(fun,}
\NormalTok{           previous,}
           \AttributeTok{step\_size =} \DecValTok{1}\NormalTok{,}
           \AttributeTok{optimizer =} \StringTok{"identity"}\NormalTok{,}
           \AttributeTok{backtracking =}\NormalTok{ T,}
           \AttributeTok{tol =} \FloatTok{10e{-}6}\NormalTok{) \{}
    \CommentTok{\#take previous gradient and a updated hessian, return update gradient with}
    \CommentTok{\#backtracking}
\NormalTok{    trial }\OtherTok{=} \DecValTok{0}
    
\NormalTok{    optimizer }\OtherTok{=} \FunctionTok{c}\NormalTok{(}\StringTok{"identity"}\NormalTok{,}\StringTok{"BFGS"}\NormalTok{)}
\NormalTok{    hessian }\OtherTok{=} \FunctionTok{fun}\NormalTok{(previous)}\SpecialCharTok{$}\NormalTok{hessian}
\NormalTok{    gradient }\OtherTok{=} \FunctionTok{fun}\NormalTok{(previous)}\SpecialCharTok{$}\NormalTok{gradient}
    
    \CommentTok{\#updating}
\NormalTok{    cur }\OtherTok{=}\NormalTok{ previous }\SpecialCharTok{{-}}\NormalTok{ step\_size }\SpecialCharTok{*} \FunctionTok{solve}\NormalTok{(hessian) }\SpecialCharTok{\%*\%}\NormalTok{ gradient}
    \CommentTok{\#backtracking}
    \ControlFlowTok{while}\NormalTok{ (backtracking }\SpecialCharTok{\&}\NormalTok{ cur }\SpecialCharTok{{-}}\NormalTok{ previous }\SpecialCharTok{\textless{}} \DecValTok{0} \SpecialCharTok{\&}\NormalTok{ trial }\SpecialCharTok{\textless{}} \DecValTok{500}\NormalTok{) \{}
\NormalTok{      step\_size }\OtherTok{=}\NormalTok{ step\_size }\SpecialCharTok{/} \DecValTok{2}
      
\NormalTok{      trial }\OtherTok{=}\NormalTok{ trial }\SpecialCharTok{+} \DecValTok{1} \CommentTok{\# avoild dead loops}
      
\NormalTok{      cur }\OtherTok{=}\NormalTok{ previous }\SpecialCharTok{{-}}\NormalTok{ step\_size }\SpecialCharTok{*} \FunctionTok{solve}\NormalTok{(hessian) }\SpecialCharTok{\%*\%}\NormalTok{ previous}
\NormalTok{    \}}
    
    \FunctionTok{return}\NormalTok{(cur)}
\NormalTok{  \}}

\NormalTok{naive\_newton }\OtherTok{=}
  \ControlFlowTok{function}\NormalTok{(fun,}
           \CommentTok{\# a function return $gradient and $hessian,}
           \AttributeTok{init =} \DecValTok{1}\NormalTok{,}
           \AttributeTok{tol =} \FloatTok{1e{-}6}\NormalTok{,}
           \AttributeTok{maxtiter =} \DecValTok{2000}\NormalTok{,}
\NormalTok{           ...) \{}
    \ControlFlowTok{if}\NormalTok{ (}\FunctionTok{any}\NormalTok{(}\FunctionTok{is.null}\NormalTok{(}\FunctionTok{fun}\NormalTok{(init)}\SpecialCharTok{$}\NormalTok{loglink),}
            \FunctionTok{is.null}\NormalTok{(}\FunctionTok{fun}\NormalTok{(init)}\SpecialCharTok{$}\NormalTok{gradient),}
            \FunctionTok{is.null}\NormalTok{(}\FunctionTok{fun}\NormalTok{(init)}\SpecialCharTok{$}\NormalTok{hessian))) \{}
      \FunctionTok{stop}\NormalTok{(}\StringTok{"fun input must return both gradient and hessian"}\NormalTok{)}
\NormalTok{    \}}
\NormalTok{    result }\OtherTok{=} \FunctionTok{tibble}\NormalTok{()}
    
\NormalTok{    i }\OtherTok{=} \DecValTok{1}
    
\NormalTok{    f }\OtherTok{=} \FunctionTok{fun}\NormalTok{(init) }\CommentTok{\# g(x\_\{k+1\})}
    
\NormalTok{    cur }\OtherTok{=}\NormalTok{ init}
    
\NormalTok{    prevlog }\OtherTok{=} \SpecialCharTok{{-}}\ConstantTok{Inf} \CommentTok{\# \textbackslash{}nabla g(x\_\{k\})}
    
    \ControlFlowTok{while}\NormalTok{ (}\FunctionTok{abs}\NormalTok{(f}\SpecialCharTok{$}\NormalTok{loglink }\SpecialCharTok{{-}}\NormalTok{ prevlog) }\SpecialCharTok{\textgreater{}}\NormalTok{ tol }\SpecialCharTok{\&\&}\NormalTok{ i }\SpecialCharTok{\textless{}}\NormalTok{ maxtiter) \{}
\NormalTok{      i }\OtherTok{=}\NormalTok{ i }\SpecialCharTok{+} \DecValTok{1}
\NormalTok{      prev }\OtherTok{=}\NormalTok{ cur}
\NormalTok{      prevlog }\OtherTok{=}\NormalTok{ f}\SpecialCharTok{$}\NormalTok{loglink}
\NormalTok{      cur }\OtherTok{=} \FunctionTok{newton\_update}\NormalTok{(f, prev)}
\NormalTok{      f }\OtherTok{=} \FunctionTok{fun}\NormalTok{(cur)}
\NormalTok{      result }\OtherTok{=} 
        \FunctionTok{rbind}\NormalTok{(result,}\FunctionTok{list}\NormalTok{(}\AttributeTok{x\_i =}\NormalTok{ prev,}
                 \StringTok{\textasciigrave{}}\AttributeTok{g(x\_i)}\StringTok{\textasciigrave{}} \OtherTok{=}\NormalTok{ f}\SpecialCharTok{$}\NormalTok{loglink,}
                 \AttributeTok{x\_i\_1 =}\NormalTok{ cur))}
\NormalTok{    \}}
    \FunctionTok{return}\NormalTok{(result)}
\NormalTok{  \}}
\end{Highlighting}
\end{Shaded}

\begin{Shaded}
\begin{Highlighting}[]
\FunctionTok{naive\_newton}\NormalTok{(Poisson\_Y)}
\end{Highlighting}
\end{Shaded}

\hypertarget{problem-3}{%
\subsection{Problem 3:}\label{problem-3}}

The data \textit{breast-cancer.csv} have 569 row and 33 columns. The
first column \textbf{ID} lables individual breast tissue images; The
second column \textbf{Diagnonsis} indentifies if the image is coming
from cancer tissue or benign cases (M=malignant, B = benign). There are
357 benign and 212 malignant cases. The other 30 columns correspond to
mean, standard deviation and the largest values (points on the tails) of
the distributions of the following 10 features computed for the
cellnuclei;

\begin{itemize}
\item radius (mean of distances from center to points on the perimeter)
\item texture (standard deviation of gray-scale values)
\item perimeter
\item area
\item smoothness (local variation in radius lengths)
\item compactness (perimeter\^ 2 / area - 1.0)
\item concavity (severity of concave portions of the contour)
\item concave points (number of concave portions of the contour)
\item symmetry
\item fractal dimension ("coastline approximation" - 1)
\end{itemize}

The goal is to build a predictive model based on logistic regression to
facilitate cancer diagnosis;

\begin{enumerate}
\item Build a logistic model to classify the images into  malignant/benign, and write down your likelihood function, its gradient and Hessian matrix.  

\item Build a logistic-LASSO model to select features, and implement a path-wise coordinate-wise optimization algorithm to obtain a path of solutions with a sequence of descending $\lambda$'s. 


\item Write a report to summarize your findings.
\end{enumerate}

\end{document}
